\section{Danh sách công việc còn lại}\label{sec:danh-sach-cong-viec-con-lai}

\subsection{Trích xuất thuộc tính thực thể:}\label{subsec:2.-trien-khai-trich-xuat-thuoc-tinh-thuc-the:}
\begin{itemize}
    \item Giải thích rõ ràng mục đích của việc trích xuất thuộc tính thực thể và cách thức nó sẽ được sử dụng trong dự án của bạn.
    \item Mô tả các phương pháp trích xuất thuộc tính thực thể khác nhau, ví dụ như dựa trên quy tắc, học máy, hoặc kết hợp cả hai.
    \item Đánh giá hiệu quả của các phương pháp trích xuất thuộc tính thực thể khác nhau trên tập dữ liệu của bạn.
\end{itemize}


\subsection{Nâng cao hiệu suất mô hình đã tinh chỉnh, không cần API của bên thứ ba:}\label{subsec:3.-nang-cao-hieu-suat-mrc-(machine-reading-comprehension)-khong-can-api-cua-ben-thu-ba:}
\begin{itemize}
    \item Thay vì chỉ nói \("\)nâng cao hiệu suất\("\), hãy nêu rõ các kỹ thuật cụ thể sẽ được sử dụng để cải thiện hiệu suất MRC, ví dụ như:
    \begin{itemize}
        \item Sử dụng các mô hình MRC tiên tiến hơn như BERT, RoBERTa, hoặc XLNet.
        \item Áp dụng các kỹ thuật học tập có giám sát hoặc học tập bán giám sát để huấn luyện mô hình MRC.
        \item Tối ưu hóa kiến trúc mô hình MRC để phù hợp với nhiệm vụ cụ thể của bạn.
    \end{itemize}
\end{itemize}


\subsection{Khám phá mối quan hệ tự động và tự động tạo câu hỏi:}\label{subsec:4.-kham-pha-moi-quan-he-tu-ong-va-tu-ong-tao-cau-hoi:}
\begin{itemize}
    \item Giải thích cách thức hoạt động của việc khám phá mối quan hệ tự động và tự động tạo câu hỏi, và lợi ích của nó đối với dự án của bạn.
    \item Mô tả các phương pháp khác nhau để thực hiện khám phá mối quan hệ tự động và tự động tạo câu hỏi, ví dụ như dựa trên biểu đồ tri thức hoặc học máy.
    \item Đánh giá hiệu quả của các phương pháp khám phá mối quan hệ tự động và tự động tạo câu hỏi khác nhau trên tập dữ liệu của bạn.
\end{itemize}


\subsection{Đánh giá chỉ số hiệu suất của MRC:}\label{subsec:5.-anh-gia-chi-so-hieu-suat-cua-mrc:}
\begin{itemize}
    \item Liệt kê các chỉ số hiệu suất khác nhau thường được sử dụng để đánh giá mô hình MRC, ví dụ như F1 score, EM score, hoặc accuracy.
    \item Giải thích cách tính toán từng chỉ số hiệu suất và ý nghĩa của chúng.
    \item So sánh hiệu suất của mô hình MRC của bạn với các mô hình khác trên cùng nhiệm vụ bằng cách sử dụng các chỉ số hiệu suất.
\end{itemize}
Ngoài ra:
\begin{itemize}
    \item Sử dụng ngôn ngữ rõ ràng, súc tích và dễ hiểu.
    \item Tránh sử dụng biệt ngữ kỹ thuật trừu tượng mà người đọc không hiểu.
    \item Cung cấp ví dụ cụ thể để minh họa cho các điểm chính của bạn.
    \item Sử dụng biểu đồ, bảng và hình ảnh để làm cho thông tin dễ tiếp thu hơn.
    \item Đảm bảo rằng danh sách việc cần làm được sắp xếp theo thứ tự ưu tiên và có thể thực hiện được.
\end{itemize}
Bằng cách thực hiện những thay đổi này, bạn có thể tạo ra một danh sách việc cần làm rõ ràng, súc tích và thuyết phục hơn cho báo cáo luận văn của mình.


