\subsection{Phương pháp học sâu (Deep learning)}\label{subsec:phuong-phap-hoc-sau-(deep-learning)}


Mạng nơ-ron sâu (Deep Neural Networks - DNNs) (
\href{https://www.sciencedirect.com/science/article/pii/S2667305323000698#br0110}{Bengio (2009)}
) đã trở nên phổ biến trong thập
kỷ qua cho nhiều ứng dụng, bao gồm xử lý ngôn ngữ tự nhiên (NLP), thị giác máy tính (CV) và nhận
dạng giọng nói. Kiến trúc DNN dựa trên Mạng nơ-ron tích chập (Convolutional Neural Network - CNN),
Mạng nơ-ron hồi quy (Recurrent Neural Network - RNN), Bộ nhớ dài hạn (Long Short-Term Memory - LSTM),
Gated Recurring Unit (GRU), Bộ mã hóa song hướng từ Transformers (BERT) được sử dụng cho các nhiệm
vụ NLP khác nhau chẳng hạn như đánh nhãn từ tính (POS tagging) (
\href{https://www.sciencedirect.com/science/article/pii/S2667305323000698#br0410}{Gopalakrishnan et al. (2019)},
\href{https://www.sciencedirect.com/science/article/pii/S2667305323000698#br1300}{Srivastava et al. (2018)},
\href{https://www.sciencedirect.com/science/article/pii/S2667305323000698#br0650}{Kumar et al. (2018)}
), dịch máy (
\href{https://www.sciencedirect.com/science/article/pii/S2667305323000698#br1470}{Wu et al. (2016)},
\href{https://www.sciencedirect.com/science/article/pii/S2667305323000698#br1430}{Wang et al. (2019)},
\href{https://www.sciencedirect.com/science/article/pii/S2667305323000698#br0400}{Gehring et al. (2017)}
), trả lời câu hỏi (
\href{https://www.sciencedirect.com/science/article/pii/S2667305323000698#br1680}{Zhu et al. (2018)},
\href{https://www.sciencedirect.com/science/article/pii/S2667305323000698#br0680}{Lei et al. (2018)},
\href{https://www.sciencedirect.com/science/article/pii/S2667305323000698#br0760}{Liu et al. (2020)}
),
gắn nhãn vai trò ngữ nghĩa (semantic role labelling) (
\href{https://www.sciencedirect.com/science/article/pii/S2667305323000698#br0480}{He et al. (2017)},
\href{https://www.sciencedirect.com/science/article/pii/S2667305323000698#br1660}{Zhou and Xu (2015)},
\href{https://www.sciencedirect.com/science/article/pii/S2667305323000698#br0800}{Marcheggiani and Titov (2017)}
), sinh hội thoại (
\href{https://www.sciencedirect.com/science/article/pii/S2667305323000698#br0700}{Li et al. (2016)},
\href{https://www.sciencedirect.com/science/article/pii/S2667305323000698#br0900}{Miranda and Kessaci (2020)},
\href{https://www.sciencedirect.com/science/article/pii/S2667305323000698#br1630}{Zhao and Eskenazi (2016)}
), sinh văn bản (
\href{https://www.sciencedirect.com/science/article/pii/S2667305323000698#br0240}{Chen et al. (2020)},
\href{https://www.sciencedirect.com/science/article/pii/S2667305323000698#br1120}{Raffel et al. (2020)},
\href{https://www.sciencedirect.com/science/article/pii/S2667305323000698#br0710}{Li et al. (2021)}
),
phân tích định hướng (sentiment analysis) (
\href{https://www.sciencedirect.com/science/article/pii/S2667305323000698#br1510}{Yu et al. (2019)},
\href{https://www.sciencedirect.com/science/article/pii/S2667305323000698#br0020}{AL-Smadi et al. (2019)},
\href{https://www.sciencedirect.com/science/article/pii/S2667305323000698#br1640}{Zhao et al. (2018)}
), tóm tắt tự động (
\href{https://www.sciencedirect.com/science/article/pii/S2667305323000698#br1600}{Zhang et al. (2019)},
\href{https://www.sciencedirect.com/science/article/pii/S2667305323000698#br1230}{See et al. (2017)},
\href{https://www.sciencedirect.com/science/article/pii/S2667305323000698#br0770}{Liu and Lapata (2019)},
\href{https://www.sciencedirect.com/science/article/pii/S2667305323000698#br0970}{Nallapati et al. (2016)},
\href{https://www.sciencedirect.com/science/article/pii/S2667305323000698#br0180}{Cai et al. (2019)}
) v.v.

\subsubsection{CNN based methods}

\begin{singlespace}
Các công trình ban đầu về trích xuất quan hệ sử dụng học sâu dựa trên mô hình học có giám sát
với tập dữ liệu huấn luyện được dán nhãn thủ công. Mô hình coi nhiệm vụ RE là vấn đề phân loại
đa lớp, trong đó mô hình gán một lớp quan hệ cho một câu chứa cặp thực thể được đề cập.
\href{https://scholar.google.com/scholar_lookup?title=Convolution%20neural%20network%20for%20relation%20extraction&publication_year=2013&author=C.%20Liu&author=W.%20Sun&author=W.%20Chao&author=W.%20Che}{Liu et al. (2013)} \textbf{đề xuất một mô hình CNN đơn giản để trích xuất quan hệ}.
Liu et al. (2013) là một trong những nhóm nghiên cứu đầu tiên sử dụng kiến trúc dựa trên Mạng nơ-ron tích chập (CNN) cho trích xuất quan hệ. Mô hình này xây dựng một kiến trúc mạng nơ-ron đầu cuối (end-to-end) với ba khối chính: lớp đầu vào, lớp tích chập và lớp mạng nơ-ron cổ điển.
\begin{itemize}
\item Lớp đầu vào: Sử dụng bảng tra cứu (lookup table) để chuyển đổi các câu đầu vào thành vector từ (word vector) bằng cách tận dụng các đặc trưng từ vựng và từ điển đồng nghĩa.
\item Lớp tích chập: Sử dụng một kernel tuần tự, ánh xạ các vector từ của lớp đầu vào vào một không gian vector mới.
\item Lớp mạng nơ-ron: Kết quả đầu ra của lớp tích chập được đưa vào mạng nơ-ron với hàm softmax để tính toán xác suất phân loại.
\end{itemize}
Trên tập dữ liệu ACE, mô hình này đạt được điểm F1 là 83,8%, vượt qua mô hình tiên tiến nhất thời bấy giờ (phương pháp Typed Dependency Kernel của Reichartz et al. (2010)).
\end{singlespace}
































\subsubsection{RNN and LSTM based methods}
\subsubsection{Encoder-decoder/transformer based methods}
